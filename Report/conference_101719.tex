\documentclass[conference]{IEEEtran}
\IEEEoverridecommandlockouts
% The preceding line is only needed to identify funding in the first footnote. If that is unneeded, please comment it out.
\usepackage{cite}
\usepackage{amsmath,amssymb,amsfonts}
\usepackage{algorithmic}
\usepackage{graphicx}
\usepackage{textcomp}
\usepackage{xcolor}
\def\BibTeX{{\rm B\kern-.05em{\sc i\kern-.025em b}\kern-.08em
    T\kern-.1667em\lower.7ex\hbox{E}\kern-.125emX}}
\begin{document}

\title{Formal Verification of Integer Multiplier Circuits\\}

\author{
\IEEEauthorblockN{Jonathan Wang}
\IEEEauthorblockA{\textit{Price College of Engineering} \\
\textit{University of Utah}\\
Salt Lake City, Utah \\
u1306458@utah.edu}
\and
\IEEEauthorblockN{Henry Silverman}
\IEEEauthorblockA{\textit{Price College of Engineering} \\
\textit{University of Utah}\\
Salt Lake City, Utah \\
henry.silverman@utah.edu}
\and
\IEEEauthorblockN{Garrett Slack}
\IEEEauthorblockA{\textit{Price College of Engineering} \\
\textit{University of Utah}\\
Salt Lake City, Utah \\
u1315263@utah.edu}
\and
\IEEEauthorblockN{Dmitry Panin}
\IEEEauthorblockA{\textit{Price College of Engineering} \\
\textit{University of Utah}\\
Salt Lake City, Utah \\
dmitry.panin@utah.edu}
}

\maketitle

\begin{abstract}
This paper employs advanced mathematical techniques, specifically polynomials, and ideals, 
to rigorously verify the correctness of an integer multiplier circuit. By leveraging algebraic methods, 
this approach will provide a deeper understanding of the circuit's behavior and enable a more robust 
verification process. 
\end{abstract}

\begin{IEEEkeywords}
polynomials, ideals, multiplier circuit
\end{IEEEkeywords}

\section{Introduction}
In the field of digital circuit design, integer multiplier circuits stand as fundamental components with critical applications in various computing 
systems, ranging from embedded devices to high-performance computing architectures. The accurate and efficient operation of these multiplier 
circuits is crucial for ensuring the reliability and correctness of arithmetic computations. As technology advances and design complexities 
escalate, the need for rigorous verification methodologies has become increasingly pronounced to guarantee the integrity of digital circuits. Our 
project tests and verifies an these integer multiplier circuits by employing the various methods studied in class on smaller circuits and circuits over
fields.

\section{Approach}
Ideal membership testing is a great method to test and verify of a 2-bit, 3-bit, 16-bit, and 32-bit integer multiplier circuit.
We manually designed the 2-bit and 3-bit multiplier from scratch and generated the larger circuits utilizing the ABC synthesis tool. Next we
convert the blif files to sing files using our parsing python script. Then derived the RTTO (Reverse Topological Term Order) from the circuit, and used it to
represent the minimal Gröbner Basis. The circuit is verified if the Gröbner Basis of $J$ + $J_{0}$ equals 0.

\section{Algorithms and Techniques}
Techniques for ideal membership testing: 
\begin{enumerate}
    \item Setup the verification formulation over the polynomial ring R. 
    \item Declare a specification polynomial from the circuit $f_{spec}$.
    \item Derive the polynomials from the gates of the circuit \{$f_{1}$, ..., $f_{s}$\}. 
    \item Set ideal J = $f_{1}$, ..., $f_{s}$. 
    % what is J0 in our case? 
    \item Create the ideal of vanishing polynomials for each variable J0 = 
    \item Take the Gröbner Basis of J + J0 = G.  
    \item Divide $f_{spec}$ by G and see if the remainder is 0. 
    \item It is a member if it's 0. 
\end{enumerate}
The steps to run Singular and abc are in the README. 

\section{Software implementations}
We implemented a Python function with assistance from ChatGPT to convert blif files to sing files. The code defines a gate mapping dictionary that maps gate types 
from the blif file to their corresponding names in Singular. It iterates through and searches for lines starting with '.gate' and extracts the gate type and its 
inputs, parsing and separating them accordingly into the Singular file. Additionally, we added code for error handling because the Singular file was empty when 
testing the code.   

\section{Concepts learned}
- Mathematical Ideals
- Gröbner Basis
- .blif to .sing conversion
- Ideal Membership testing
- Miters

\section{Labour Division}
Jonathan: a bit \\
Henry: absolutely nothing -- JDubs \\
Garrett: a bit \\
Dmitri: a bit

\begin{thebibliography}{00}
\bibitem{b1} D. Ritirc, A. Biere and M. Kauers, "Column-wise verification of multipliers using computer algebra," 2017 Formal Methods in Computer Aided Design (FMCAD), Vienna, Austria, 2017, pp. 23-30, doi: 10.23919/FMCAD.2017.8102237.
\bibitem{b2} T. Pruss, P. Kalla and F. Enescu, "Efficient Symbolic Computation for Word-Level Abstraction From Combinational Circuits for Verification Over Finite Fields," in IEEE Transactions on Computer-Aided Design of Integrated Circuits and Systems, vol. 35, no. 7, pp. 1206-1218, July 2016, doi: 10.1109/TCAD.2015.2501301.
\end{thebibliography}
\vspace{12pt}

\end{document}
